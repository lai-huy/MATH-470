\documentclass[12pt]{article}

\usepackage[affil-it]{authblk}
\usepackage[shortlabels]{enumitem}
\usepackage[utf8]{inputenc}
\usepackage{algorithm, algorithmicx, algpseudocode}
\usepackage{amsfonts, amsthm, amsmath, amssymb}
\usepackage{color}
\usepackage{cancel, textcomp}
\usepackage{enumerate}
\usepackage[mathscr]{euscript}
\usepackage{fancyhdr, fancyvrb}
\usepackage{fullpage}
\usepackage[left=0.5in,right=0.5in,headsep=0.5in,headheight=0.5in]{geometry}
\usepackage{graphicx}
\usepackage{hyperref}
\usepackage{latexsym}
\usepackage{mathtools}
\usepackage{minted}
\usepackage{times}
\usepackage{xcolor}
\usepackage{physics}
\usepackage{tikz-cd}
\usepackage[warnunknown, fasterrors, mathletters]{ucs}
\usepackage[nointegrals]{wasysym}

\newcommand{\hw}[2]{
    \noindent
    \begin{center}
        \framebox{
            \vbox{
                \hbox to 7in { {\bf MATH 470: Communications and Cryptography } \hfill  }
                \vspace{2mm}
                \hbox to 7in { {\Large \hfill Homework #1\hfill} }
                \vspace{2mm}
                \hbox to 7in { {\it Due date: #2 \hfill Name: Huy Lai } }
            }
        }
    \end{center}
    \vspace*{4mm}
}

\newcounter{prob}
\setcounter{prob}{0}
\newcounter{subprob}
\setcounter{subprob}{0}

\newcommand{\problem}{\setcounter{subprob}{0}\stepcounter{prob}{\noindent\textbf{Problem \theprob.}}\ }
\newcommand{\subproblem}{\stepcounter{subprob}{\noindent\textbf{Subproblem \thesubprob.}}\ }
\newcommand{\solution}{\noindent\textbf{Solution:}\newline}

\newcommand{\babc}{\begin{enumerate}[a)]}
\newcommand{\eabc}{\end{enumerate}}

\everymath{\displaystyle}

\setlength{\parskip}{.1in}
\setlength{\headheight}{15pt}
\setlength{\topmargin}{0pt}

\fancyhf{}
\pagestyle{fancy}
\lhead{MATH 470: Communications and Cryptography}
\rhead{Texas A\&M University}
\cfoot{\thepage}


\begin{document}
\hw{3}{13 September 2023}

\problem Let $p=587$ and numbers $a=345$, compute $a^{-1}\mod{p}$ in two
ways:

\begin{enumerate}[(i)]
    \item Use the extended Euclidean algorithm.
    \item Use the fast power algorithm and Fermat’s little theorem. 
\end{enumerate}

\solution Using the Extended Euclidean Algorithm

\noindent
\begin{tabular}{|c|c|c|c|}
    \hline
    $q$ & $r$   & $u$  & $v$  \\
    \hline
        & $587$ & $0$   & $1$   \\
        & $345$ & $1$   & $0$   \\
    $1$ & $242$ & $-1$  & $1$   \\
    $1$ & $103$ & $2$   & $-1$  \\
    $2$ & $36$  & $-5$  & $3$   \\
    $2$ & $31$  & $12$  & $-7$  \\
    $1$ & $5$   & $-17$ & $10$  \\
    $6$ & $1$   & $114$ & $-67$ \\
    $5$ & $0$   & $u$   & $v$   \\
    \hline
\end{tabular}

\noindent
According to the EEA, the integers $(u,v)=(114,-67)$.\\
$354(114)+587(-67)=1$\\
$345^{-1}\equiv114\mod{587}$

\newpage
\noindent
Using the fast power algorithm and Fermat's Little Theorem.

\noindent
Fermat's Little Theorem states
\[a^{p-1}\equiv 1\mod{p}\]
Multiplying both sides of this by $a^{-1}$ results in
\[a^{p-2}\equiv a^{-1}\mod{p}\]
Therefore, 
\[a^{-1}\equiv a^{p-2}\mod{p}\]

\noindent
$p-2=585_{10}=1001001001_2$

\begin{flalign*}
a^{2^0} &\equiv a^{1}   \equiv 345\mod{p} &\\
a^{2^1} &\equiv a^{2}   \equiv 451\mod{p} &\\
a^{2^2} &\equiv a^{4}   \equiv 299\mod{p} &\\
a^{2^3} &\equiv a^{8}   \equiv 177\mod{p} &\\
a^{2^4} &\equiv a^{16}  \equiv 218\mod{p} &\\
a^{2^5} &\equiv a^{32}  \equiv 564\mod{p} &\\
a^{2^6} &\equiv a^{64}  \equiv 529\mod{p} &\\
a^{2^7} &\equiv a^{128} \equiv 429\mod{p} &\\
a^{2^8} &\equiv a^{256} \equiv 310\mod{p} &\\
a^{2^9} &\equiv a^{512} \equiv 419\mod{p} &\\
\end{flalign*}
$345^{585}=345^{2^9+2^6+2^3+2^0}\equiv419\cdot529\cdot177\cdot345\equiv114\mod{587}$\\
$345^{-1}\equiv114\mod{587}$

\newpage
\problem Let $p$ be a prime and let $q$ be a prime that divides $p-1$. Let $a\in\mathbb{F}_p^*$ and let $b=a^{\frac{p-1}{q}}$. Prove that either $b=1$ or else $b$ has order $q$.

\solution The proof is as follows.
\begin{proof}
Let $k$ be the order of $b$.\\
Raising the definition of $b$ to the power of $q$ will result in
\[b^q=a^{p-1}\]
By Fermat's Little Theorem, $b^q\equiv a^{p-1}\equiv 1\mod{p}$\\
By Proposition 1.29 in the Textbook, $k\mid q$. Since $q$ is prime, then $k=1$ or $k=q$.\\
As a result either $b$ has order $q$, or it has order 1.
\end{proof}

\problem Let $p$ be a prime such that $q=\frac{1}{2}(p-1)$ is also prime. Suppose that $g$ is an integer satisfying 
\[g\not\equiv 0\mod{p},g\not\equiv\pm1\mod{p},g^q\not\equiv1\mod{p}\]
Prove that $g$ is a primitive root modulo $p$.

\solution The proof is as follows.
\begin{proof}
Let $k$ be the order of $g$. Then by proposition, $k\mid (p-1)$.\\
Since $p-1=2q$ with $q$ prime, this means that
\[k=1\text{ or } k=2 \text{ or } k=q \text{ or } k=2q\]

\noindent
Since $g\not\equiv\pm1\mod{p}$, then $k\neq1$ and $k\neq2$.\\
Since $g^q\not\equiv1\mod{p}$, then $k\neq q$.\\
As a result $k=2q\Rightarrow k=p-1$.\\
With this, the order of $g$ is $p-1$, this satisfies the definition of a primitive root.
Therefore, $g$ is a primitive root modulo $p$.
\end{proof}

\newpage
\problem Let $p$ be an odd prime number and let $b$ be an integer with $p\nmid b$. Prove that either $b$ has two square roots modulo $p$ or else $b$ has no square roots modulo $p$. In other words, prove that the congruence
\[X^2\equiv b\mod{p}\]
has either two solutions or no solutions in $\mathbb{Z}/p\mathbb{Z}$. (What happens for $p=2$? What happens if $p\mid b$?)

\solution The proof is as follows.
\begin{proof}
Let $a_1,a_2$ be solutions to the congruency.\\
Then by proposition of modulo, $p\mid\left(a_1^2-b\right)$ and $p\mid\left(a_2^2-b\right)$.\\
By proposition of divisibility, $p\mid\left[\left(a_1^2-b\right)-\left(a_2^2-b\right)\right]$\\
This can be rewritten as
\[p\mid(a_1-a_2)(a_1+a_2)\]

\noindent
From this, $p$ must divide either $a_1-a_2$ or $a_1+a_2$.\\
If $p\mid(a_1-a_2)$, then $a_1\equiv a_2\mod{p}$. If $p\mid(a_1+a_2)$, then $a_1\equiv-a_2\mod{p}$.\\
As a result, there are a most two solutions.

\noindent
We prove that one solution is not possible by contradiction.\\
Let $a$ be the solution to the congruency.\\
Then, $a^2\equiv b\mod{p}$.\\
We can generate another unique solution by completing the square as follows
\[p^2+2ap+a^2\equiv b\mod{p}\]
This gives that $(p+a)^2\equiv b\mod{p}$. However, $p+a\not=a$.\\
Therefore, if one solution exists, another can be found.\\
This proves that there are zero or two solutions.
\end{proof}

\clearpage
\problem Problem 5

\subproblem Let $p=13$ and let $g=2$. Note that $p$ is prime and that $g$ is a primitive root modulo $p$. Make a list of the powers of $g$ and their orders modulo $p$ (i.e., for each $a\in\{1,2,3,\cdots,12\}$, write down $g^a \mod{p}$ and the order of $g^a \mod{p}$). What are all the primitive roots modulo $p$? Compute $\phi(p-1)$, where $\phi$ is the Euler’s totient function.

\solution
\begin{tabular}{|c|c|c|}
\hline
$a$ & $2^a \mod{13}$ & Order \\
\hline
1 & 2 & 12 \\
2 & 4 & 6 \\
3 & 8 & 4 \\
4 & 3 & 3 \\
5 & 6 & 2 \\
6 & 12 & 2 \\
7 & 11 & 12 \\
8 & 9 & 3 \\
9 & 5 & 6 \\
10 & 10 & 12 \\
11 & 7 & 12 \\
12 & 1 & 1 \\
\hline
\end{tabular}

\noindent
The primitive roots modulo $p$ are:
\begin{flalign*}
2^{1}\mod{p}    &= 2  & 2^{7}\mod{p}    &= 11 & \\
2^{10}\mod{p}   &= 10 & 2^{11}\mod{p}   &= 7  & \\
\end{flalign*}
$\phi(n-1)=\phi(12)=4$

\clearpage
\subproblem Let $p$ be a prime, let $g$ be a primitive root modulo $p$, and let $a$ be an integer. Prove that the order of $g^a\mod{p}$ is exactly $\frac{p-1}{\gcd(a,p-1)}$. Explain why this implies that the number of primitive roots modulo $p$ is exactly $\phi(p-1)$, assuming that a primitive root $g$ modulo $p$ exists. (Looking over your work in part (a) may help you gain some intuition).

\solution
Let $n=\frac{p-1}{\gcd(a,p-1)}$
To prove that the order of $g^a\mod{p}$ is exactly $n$ we must show that the order cannot be smaller than this number and that the order cannot be larger than this number.

\noindent
First prove that the order cannot be smaller

\begin{proof}
Let $k$ be the order of $g^a\mod{p}$\\
This means that $k$ is the smallest positive integer such that $\left(g^a\right)^k \equiv 1\mod{p}$.\\
Assume that $k<n$.

\noindent
By Fermat's Little Theorem, $g^{p-1}\equiv1\mod{p}$.\\
Raising both sides by the power of $n$ results in
\[\left(g^{p-1}\right)^n\equiv1^n\equiv1\mod{p}\]

\noindent
Consider
\[\left(g^a\right)^n\equiv g^{an}\mod{p}\]
Multiplying the exponent by $\frac{p-1}{p-1}$ results in
\[g^{an}\equiv g^{(p-1)\cdot\frac{an}{p-1}}\mod{p}\]
Using Fermat's Little Theorem gives
\[g^{an}\equiv 1^{\frac{an}{p-1}}\mod{p}\]
This implies that the order of $g^a\mod{p}$ is at most $n$.\\
This contriducts with the assumption that the order is smaller
\end{proof}

\clearpage
\problem You may assume that the following integers $p$ and $q$ are primes:
\begin{flalign*}
p &= 1234567890123456789012345678901234567890123456789012345678901234567890123459287 \\
q &= 617283945061728394506172839450617283945061728394506172839450617283945061729643 \\
\end{flalign*}
Also note that $p=2q+1$. Find the smallest positive integer $g$ that is a primitive root modulo $p$.

\solution
\inputminted{py}{proot.py}

\begin{figure}[!ht]
    \centering
    \includegraphics[width=0.9\textwidth]{Question 6.png}
    \caption{Output}
\end{figure}

\end{document}
