\documentclass[12pt]{article}

\usepackage[affil-it]{authblk}
\usepackage[shortlabels]{enumitem}
\usepackage[utf8]{inputenc}
\usepackage{algorithm, algorithmicx, algpseudocode}
\usepackage{amsfonts, amsthm, amsmath, amssymb}
\usepackage{color}
\usepackage{cancel, textcomp}
\usepackage{enumerate}
\usepackage[mathscr]{euscript}
\usepackage{fancyhdr, fancyvrb}
\usepackage{fullpage}
\usepackage[left=0.5in,right=0.5in,headsep=0.5in,headheight=0.5in]{geometry}
\usepackage{graphicx}
\usepackage{hyperref}
\usepackage{latexsym}
\usepackage{mathtools}
\usepackage{minted}
\usepackage{times}
\usepackage{xcolor}
\usepackage{physics}
\usepackage{tikz-cd}
\usepackage[warnunknown, fasterrors, mathletters]{ucs}
\usepackage[nointegrals]{wasysym}

\newcommand{\hw}[2]{
    \noindent
    \begin{center}
        \framebox{
            \vbox{
                \hbox to 7in { {\bf MATH 470: Communications and Cryptography } \hfill  }
                \vspace{2mm}
                \hbox to 7in { {\Large \hfill Homework #1\hfill} }
                \vspace{2mm}
                \hbox to 7in { {\it Due date: #2 \hfill Name: Huy Lai } }
            }
        }
    \end{center}
    \vspace*{4mm}
}

\newcounter{prob}
\setcounter{prob}{0}
\newcounter{subprob}
\setcounter{subprob}{0}

\newcommand{\problem}{\setcounter{subprob}{0}\stepcounter{prob}{\noindent\textbf{Problem \theprob.}}\ }
\newcommand{\subproblem}{\stepcounter{subprob}{\noindent\textbf{Subproblem \thesubprob.}}\ }
\newcommand{\solution}{\noindent\textbf{Solution:}\newline}

\newcommand{\babc}{\begin{enumerate}[a)]}
\newcommand{\eabc}{\end{enumerate}}

\everymath{\displaystyle}

\setlength{\parskip}{.1in}
\setlength{\headheight}{15pt}
\setlength{\topmargin}{0pt}

\fancyhf{}
\pagestyle{fancy}
\lhead{MATH 470: Communications and Cryptography}
\rhead{Texas A\&M University}
\cfoot{\thepage}


\begin{document}
\thispagestyle{empty}
\hw{5}{4 October 2023}

\problem Alice publishes her RSA public key: modulus $N=2038667$ and exponent $e=103$.\\
\subproblem Bob wants to send Alice the message $m=892383$. What ciphertext does Bob send to Alice?

\solution
Bob sends $c=m^e\equiv45293\mod{N}$

\subproblem Alice knows that her modulus factors into a product of two primes, one of which is $p=1301$. Find a decryption exponent $d$ for Alice.

\solution
The modulus $N=1301\cdot1567$, so $\phi(N)=1300\cdot1568=2035800$.\\
A decryption exponent is given by a solution to
\[e\cdot d\equiv 1\mod{\phi(N)}\]
The solution is $d=810367\mod{\phi(N)}$

\subproblem Alice receives the ciphertext $c=317730$ from Bob. Decrypt the message.

\solution
Alice needs to solve $m^e\equiv c\mod{N}$.\\
Raising both sides to the power of $d$ yields
\[m\equiv c^{d}\mod{N}\equiv 514407\mod{N}\]

\newpage
\problem Let $N=pq=352717$ and $(p-1)(q-1)=351520$, use the method described in Remark 3.11 to determine $p$ and $q$.

\solution
$p+q=N+1-(p-1)(q-1)=1198$, so
\[X^2-(p+q)X+N=X^2-1198X+352717=(X-677)(X-521)\]
Hence $p=677,q=521$

\problem Alice decides to use RSA with the public key $N=1889570071$. In order to guard against transmission errors, Alice has Bob encrypt his message twice, once using the encryption exponent $e_1= 1021763679$ and once using the encryption exponent $e_2=519424709$. Eve intercepts the two encrypted messages
\[c_1=1244183534 \text{ and } c_2=732959706\]
Assuming that Eve also knows $N$ and the two encryption exponents $e_1$ and $e_2$, use the method described in Example 3.15 to help Eve recover Bob’s plaintext without finding a factorization of $N$.

\solution
With the method described in Example 3.15, we find that
\[u\cdot c_1+v\cdot c_2=1\]
with
\[u= 252426389 \text{ and } v=-496549570\]
Then the plaintext is
\[m\equiv c_1^{u}\cdot c_2^{v}\equiv 105459238\mod{N}\]

\newpage
\problem Use the Miller–Rabin test on each of the following numbers. In each case, either provide a Miller-Rabin witness for the compositeness of $n$, or conclude that $n$ is probably prime by providing 10 numbers that are not Miller-Rabin witnesses for $n$.

\subproblem $n=118901509$

\solution
$n-1=118901508=2^2\cdot29725377$
\begin{flalign*}
    2^{29725377}        & \equiv 7906806 \mod{n} & \\
    2^{2\cdot29725377}  & \equiv -1      \mod{n} & \\
    3^{29725377}        & \equiv -1      \mod{n} & \\
    3^{2\cdot29725377}  & \equiv 1       \mod{n} & \\
    5^{29725377}        & \equiv -1      \mod{n} & \\
    5^{2\cdot29725377}  & \equiv 1       \mod{n} & \\
    7^{29725377}        & \equiv 7906806 \mod{n} & \\
    7^{2\cdot29725377}  & \equiv -1      \mod{n} & \\
    11^{29725377}       & \equiv -1      \mod{n} & \\
    11^{2\cdot29725377} & \equiv 1       \mod{n} & \\
\end{flalign*}
Thus $2,3,5,7$, and $11$ are not Miller–Rabin witnesses for $n$. $n$ is probably prime.

\subproblem $n=118901521$

\solution
$n-1=118901520=2^4\cdot7431345$
\begin{flalign*}
    2^{7431345}       & \equiv45274074\mod{n} & \\
    2^{2\cdot7431345} & \equiv1758249\mod{n}  & \\
    2^{4\cdot7431345} & \equiv1\mod{n}        & \\
    2^{8\cdot7431345} & \equiv1\mod{n}        & \\
\end{flalign*}
Thus 118901521 is composite. It factors into $n=271\cdot541\cdot811$

\newpage
\problem Show that the Elgamal encryption protocol is insecure against a Chosen Ciphertext Attack. More specifically, suppose Bob has published a prime $p$, primitive root $g\mod{p}$, and his public key $B$. Alice has sent Bob a ciphertext $(c_1,c_2)$. So far Eve only knows $p,g,B$, and $(c_1,c_2)$. But suppose now that Eve can somehow make Bob decrypt “random-looking” ciphertexts $(c_1', c_2')$ of Eve’s choice (by “random-looking” we mean that Bob should not be able to tell that $(c_1', c_2')$ or its decryption is related to Alice’s message in any way). Show how Eve can use this ability to decrypt Alice’s message.

\solution
We can generate a ``random" cipher text for Bob to decrypt using a second message $m'$ and Eve's secret key $k'$ as follows:
\begin{flalign*}
    c_1' & = c_1\cdot g^{k'}\equiv g^{k+k'}\mod{p}                         & \\
    c_2' & = c_2\cdot B^{k'}\cdot m'\equiv(m\cdot m')\cdot B^{k+k'}\mod{p} & \\
\end{flalign*}

\noindent
Bob uses this information to calculate the ``encrypted" message to send back as follows:
\[m''=m\cdot m'\]
With this, the original message can be recovered by calculating $m''\cdot(m')^{-1}$.

\problem Let $N=pq$ be a product of two distinct odd primes $p$ and $q$. Show that there are four square roots of $1$ modulo $N$. In other words, show that there are exactly four integers in $\{1,2,3,\cdots,N-1\}$ whose squares are congruent to $1\mod{N}$.

\solution
Since $p-1$ and $q-1$ are both even, $\exists m,n\in\mathbb{Z}$ such that $p-1=2m,q-1=2n$\\
By Fermat's Little Theorem we know that if $a\nmid p$ and $p$ is prime,
\[a^{p-1}\equiv 1\mod{p}\]
\noindent
Substituting $2m$ for $p-1$ gives us
\[a^{2m}\equiv 1\mod{p}\]
This is equivalent to $(a^m)^2\equiv1\mod{p}$\\
We know from a previous homework question that there exists exactly two solutions to the above equation.\\
A similar logic can be applied to $q$.\\
As a result, there are exactly four solutions.

\newpage
\problem Suppose that you are given an integer $N$ and a pair of integers $e,d$ with the promise that $N$ is the product of two large primes and that $ed\equiv1\mod{\phi(N)}$ (but you are not given the factors of $N$ nor the value of $\phi(N)$. Describe an algorithm that efficiently factors $N$.

\solution The algorithm is as follows

\begin{enumerate}
    \item Compute a random integer a such that $1<a<N$. This is similar to how the Miller-Rabin primality test selects random witnesses.
    \item Calculate the value $x\equiv a^d\mod{N}$. Since $ed\equiv1\mod{\phi{(N)}}$, this means that $x^e\equiv a^{(d\cdot e)}\equiv a^{(k\cdot\phi(N)+1)}\equiv a\mod{N}$, where $k$ is an integer.

    \item Use the Extended Euclidean Algorithm to find the $\gcd(N,x-a)$. If the GCD is greater than 1, then it means that $N$ has a non-trivial factor in common with $x-a$.

    \item If the GCD is 1, repeat steps 1-3 with a different random value of a. Keep doing this until you find a GCD greater than 1 or until you've tried a sufficient number of random values of a.

    \item Once you find a GCD greater than 1 (let's say it's G), you have effectively found one of the prime factors of N, either p or q.
\end{enumerate}
\end{document}