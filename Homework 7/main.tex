\documentclass[12pt]{article}

\usepackage[affil-it]{authblk}
\usepackage[shortlabels]{enumitem}
\usepackage[utf8]{inputenc}
\usepackage{algorithm, algorithmicx, algpseudocode}
\usepackage{amsfonts, amsthm, amsmath, amssymb}
\usepackage{color}
\usepackage{cancel, textcomp}
\usepackage{enumerate}
\usepackage[mathscr]{euscript}
\usepackage{fancyhdr, fancyvrb}
\usepackage{fullpage}
\usepackage[left=0.5in,right=0.5in,headsep=0.5in,headheight=0.5in]{geometry}
\usepackage{graphicx}
\usepackage{hyperref}
\usepackage{latexsym}
\usepackage{mathtools}
\usepackage{minted}
\usepackage{times}
\usepackage{xcolor}
\usepackage{physics}
\usepackage{tikz-cd}
\usepackage[warnunknown, fasterrors, mathletters]{ucs}
\usepackage[nointegrals]{wasysym}

\newcommand{\hw}[2]{
    \noindent
    \begin{center}
        \framebox{
            \vbox{
                \hbox to 7in { {\bf MATH 470: Communications and Cryptography } \hfill  }
                \vspace{2mm}
                \hbox to 7in { {\Large \hfill Homework #1\hfill} }
                \vspace{2mm}
                \hbox to 7in { {\it Due date: #2 \hfill Name: Huy Lai } }
            }
        }
    \end{center}
    \vspace*{4mm}
}

\newcounter{prob}
\setcounter{prob}{0}
\newcounter{subprob}
\setcounter{subprob}{0}

\newcommand{\problem}{\setcounter{subprob}{0}\stepcounter{prob}{\noindent\textbf{Problem \theprob.}}\ }
\newcommand{\subproblem}{\stepcounter{subprob}{\noindent\textbf{Subproblem \thesubprob.}}\ }
\newcommand{\solution}{\noindent\textbf{Solution:}\newline}

\newcommand{\babc}{\begin{enumerate}[a)]}
\newcommand{\eabc}{\end{enumerate}}

\everymath{\displaystyle}

\setlength{\parskip}{.1in}
\setlength{\headheight}{15pt}
\setlength{\topmargin}{0pt}

\fancyhf{}
\pagestyle{fancy}
\lhead{MATH 470: Communications and Cryptography}
\rhead{Texas A\&M University}
\cfoot{\thepage}


\begin{document}
\hw{7}{18 October 2023}
\thispagestyle{empty}

\problem Illustrate the quadratic sieve, as was done in Fig. 3.3 (page 161), by sieving prime powers up to $B$ on the values of $F(T)=T^2-N$ in the indicated range. Sieve $N=493$ using prime powers up to $B=11$ on values from $F(23)$ to $F(38)$. Use the relation(s) that you find to factor $N$.

\solution The Quadratic Sieve is as follows
\begin{table}[!ht]
    \centering
    \begin{tabular}{|cccccccccccccccc|}
        \hline
        $23$             & $24$ & $25$             & $26$           & $27$             & $28$           & $29$             & $30$            & $31$             & $32$             & $33$             & $34$           & $35$             & $36$            & $37$             & $38$           \\
        \hline
        $36$             & $83$ & $132$            & $183$          & $236$            & $291$          & $348$            & $407$           & $468$            & $531$            & $596$            & $663$          & $732$            & $803$           & $876$            & $951$          \\
        $\downarrow 2$   &      & $\downarrow 2$   &                & $\downarrow 2$   &                & $\downarrow 2$   &                 & $\downarrow 2$   &                  & $\downarrow 2$   &                & $\downarrow 2$   &                 & $\downarrow 2$   &                \\
        $18$             & $83$ & $66$             & $183$          & $118$            & $291$          & $174$            & $407$           & $234$            & $531$            & $298$            & $663$          & $366$            & $803$           & $438$            & $951$          \\
        $\downarrow 3$   &      &                  & $\downarrow 3$ &                  &                & $\downarrow 3$   &                 &                  & $\downarrow 3$   &                  &                & $\downarrow 3$   &                 &                  & $\downarrow 3$ \\
        $6$              & $83$ & $66$             & $61$           & $118$            & $291$          & $58$             & $407$           & $234$            & $177$            & $298$            & $663$          & $122$            & $803$           & $438$            & $317$          \\
                         &      & $\downarrow 3$   &                &                  & $\downarrow 3$ &                  &                 & $\downarrow 3$   &                  &                  & $\downarrow 3$ &                  &                 & $\downarrow 3$   &                \\
        $6$              & $83$ & $22$             & $61$           & $118$            & $97$           & $58$             & $407$           & $78$             & $177$            & $298$            & $221$          & $122$            & $803$           & $146$            & $317$          \\
        $\downarrow 2^2$ &      & $\downarrow 2^2$ &                & $\downarrow 2^2$ &                & $\downarrow 2^2$ &                 & $\downarrow 2^2$ &                  & $\downarrow 2^2$ &                & $\downarrow 2^2$ &                 & $\downarrow 2^2$ &                \\
        $3$              & $83$ & $11$             & $61$           & $59$             & $97$           & $29$             & $407$           & $39$             & $177$            & $149$            & $221$          & $61$             & $803$           & $73$             & $317$          \\
                         &      &                  &                &                  &                &                  &                 & $\downarrow 3^2$ &                  &                  &                &                  &                 &                  &                \\
        $3$              & $83$ & $11$             & $61$           & $59$             & $97$           & $29$             & $407$           & $13$             & $177$            & $149$            & $221$          & $61$             & $803$           & $73$             & $317$          \\
        $\downarrow 3^2$ &      &                  &                &                  &                &                  &                 &                  & $\downarrow 3^2$ &                  &                &                  &                 &                  &                \\
        $1$              & $83$ & $11$             & $61$           & $59$             & $97$           & $29$             & $407$           & $13$             & $59$             & $149$            & $221$          & $61$             & $803$           & $73$             & $317$          \\
                         &      & $\downarrow 11$  &                &                  &                &                  &                 &                  &                  &                  &                &                  & $\downarrow 11$ &                  &                \\
        $1$              & $83$ & $1$              & $61$           & $59$             & $97$           & $29$             & $407$           & $13$             & $59$             & $149$            & $221$          & $61$             & $73$            & $73$             & $317$          \\
                         &      &                  &                &                  &                &                  & $\downarrow 11$ &                  &                  &                  &                &                  &                 &                  &                \\
        $1$              & $83$ & $1$              & $61$           & $59$             & $97$           & $29$             & $37$            & $13$             & $59$             & $149$            & $221$          & $61$             & $73$            & $73$             & $317$          \\
        \hline
    \end{tabular}
    \caption{Sieving $n=493$}
\end{table}

\noindent
The two values $F(23)$ and $F(25)$ have been sieved down to $1$, yielding the congruences
\[F(23)\equiv 36\equiv2^2\cdot3^2\mod{493}\quad\text{and}\quad F(25)\equiv132\equiv2^3\cdot3\cdot11\mod{493}\]

\noindent
Since $F(23)$ is itself congruent to a square, we can compute $\gcd(23-2\cdot3,493)\equiv17$ which gives the factorization $493=17\cdot29$.

\newpage
\problem Let $p$ be an odd prime and let $a$ be an integer with $p\nmid a$. Prove that $a^{\frac{p-1}{2}}\equiv 1\mod{p}$ if and only if $a$ is a quadratic residue modulo $p$.

\solution
First we prove that if $a^{\frac{p-1}{2}}\equiv 1\mod{p}$, then $a$ is a quadratic residue modulo $p$.
\begin{proof}
    Let $g$ be a primitive root modulo $p$ such that $a\equiv g^k\mod{p}$ for some integer $k$.
    \begin{flalign*}
        a^{\frac{p-1}{2}}     & \equiv 1\mod{p} & \\
        (g^k)^{\frac{p-1}{2}} & \equiv 1\mod{p}
    \end{flalign*}
    Since $g$ is a primitive root, it has order $p-1$.\\
    By proposition, $p-1\mid k\cdot\frac{p-1}{2}$ which can be rewritten as $p-1\mid \frac{k}{2}(p-1)$.\\
    This requires that $\frac{k}{2}\in\mathbb{Z}$

    \noindent
    Let $\kappa\equiv g^{\frac{k}{2}}\mod{p}$
    \begin{flalign*}
        \kappa^2 & \equiv g^{2\cdot\frac{k}{2}}\mod{p} & \\
        \kappa^2 & \equiv g^{k}\mod{p}                 & \\
        \kappa^2 & \equiv a\mod{p}
    \end{flalign*}
    From this, we have found an integer that satisfies the definition of a quadratic residue.
\end{proof}

\noindent
Next we prove that if $a$ is a quadratic residue modulo $p$, then $a^{\frac{p-1}{2}}\equiv 1\mod{p}$.
\begin{proof}
    Since $a$ is a quadratic residue, $\exists c\in\mathbb{Z}$ such that $c^2\equiv a\mod{p}$.\\
    Additionally, $p\nmid c$ since $p\mid c$ would imply that $c\equiv 0\mod{p}$ and cannot square to $a\not\equiv 0\mod{p}$.
    \begin{flalign*}
        c^2                     & \equiv a\mod{p}                 & \\
        (c^{2})^{\frac{p-1}{2}} & \equiv a^{\frac{p-1}{2}}\mod{p} & \\
        c^{p-1}                 & \equiv a^{\frac{p-1}{2}}\mod{p} & \\
    \end{flalign*}
    From this we can apply Fermat's Little Theorem to $c^{p-1}$ and get that
    \[a^{\frac{p-1}{2}}\equiv 1\mod{p}\]
\end{proof}

\newpage
\problem Let $p$ be a prime satisfying $p\equiv3\mod{4}$. Let $a$ be a quadratic residue modulo $p$. Prove that the number
\[b\equiv a^{\frac{p+1}{4}}\pmod{p}\]
has the property that $b^2\equiv a\pmod{p}$

\solution The proof is as follows

\begin{proof}
    \begin{flalign*}
        b^2 & \equiv a^{\frac{p+1}{2}}\mod{p}        & \\
            & \equiv a^{1+\frac{p-1}{2}}\mod{p}      & \\
            & \equiv a\cdot a^{\frac{p-1}{2}}\mod{p}
    \end{flalign*}
    Since $a$ is a quadratic residue, we use the proof from question 2 and get that
    \[a^{\frac{p-1}{2}}\equiv 1\mod{p}\]
    Using this fact we get
    \[b^2\equiv a\mod{p}\]
\end{proof}

\newpage
\problem Recall that $p=9907$ is a prime. Use quadratic reciprocity to compute $\left(\frac{1002}{9907}\right)$.

\solution
First note that $9907\equiv 3\mod{4}$ and $9907\equiv 3\mod{8}$
\begin{flalign*}
    \left(\frac{1002}{9907}\right) & =\left(\frac{2\cdot3\cdot167}{9907}\right)                                           & \\
                                   & =\left(\frac{2}{9907}\right)\left(\frac{3}{9907}\right)\left(\frac{167}{9907}\right) & \\
                                   & =-1\left(\frac{3}{9907}\right)\left(\frac{167}{9907}\right)                          & \\
                                   & =-1\left(-\frac{9907}{3}\right)\left(-\frac{9907}{167}\right)                        & \\
                                   & =-\left(-\frac{1}{3}\right)\left(-\frac{54}{167}\right)                              & \\
                                   & =\left(-\frac{54}{167}\right)                                                        & \\
                                   & =-\left(\frac{2\cdot27}{167}\right)                                                  & \\
                                   & =-\left(\frac{2}{167}\right)\left(\frac{27}{167}\right)                              & \\
                                   & =-\left(\frac{27}{167}\right)                                                        & \\
                                   & =-\left(\frac{167}{27}\right)                                                        & \\
                                   & =-\left(\frac{5}{27}\right)                                                          & \\
                                   & =-\left(\frac{27}{5}\right)                                                          & \\
                                   & =-\left(\frac{2}{5}\right)                                                           & \\
                                   & =-1
\end{flalign*}

\newpage
\problem Let $p$ be an odd prime. Prove that the number of quadratic residues modulo $p$ is exactly $\frac{p+1}{2}$

\solution
The integer $0$ is a trivial example of a quadratic residue so $+1$ will be added to the final number count.
\begin{proof}
    The quadratic residues of $p$ are the integers which will result from the evaluation of the squares:
    \[1^2,2^2,3^2,\cdots,(p-1)^2\mod{p}\]
    and so these $p-1$ integers fall into congruent pairs modulo $p$, namely:
    \begin{align*}
        1^2                          & \equiv (p-1)^2\mod{p}                      \\
        2^2                          & \equiv (p-2)^2\mod{p}                      \\
                                     & \vdots                                     \\
        \left(\frac{p-1}{2}\right)^2 & \equiv \left(\frac{p+1}{2}\right)^2\mod{p}
    \end{align*}

    \noindent
    Therefore, each quadratic residue modulo $p$ is congruent modulo $p$ to one of the $\frac{p-1}{2}$ integers:
    \[1^2,2^2,\dots,\left(\frac{p-1}{2}\right)^2\]

    \noindent
    All we need to do now is show that no two of these integers are congruent modulo $p$.\\
    Suppose that $r^2\equiv s^2\mod{p}$ for some $1\leq r\leq s\leq\frac{p-1}{2}$.\\
    What we need to do is prove that $r=s$.

    \noindent
    By proposition, $p\mid(r-s)(r+s)$.\\
    This means that either $p\mid(r-s)$ or $p\mid(r+s)$\\
    $p\nmid(r+s)$ as $2\leq r+s\leq p-1$.

    \noindent
    This leaves $p\mid(r-s)$\\
    As $0\leq r-s<\frac{p-1}{2}$, this condition can only happen when $r-s=0$ or equivalently, when $r=s$.

    \noindent
    Therefore, there must be exactly $\frac{p-1}{2}+1=\frac{p+1}{2}$ quadratic residues modulo $p$.
\end{proof}

\newpage
\problem My RSA public key is $(N,e)$, where
\begin{align*}
    N & = 3426473875287793756703750981622962137419589116424756456135570641437827 \\
    e & = 65537
\end{align*}
I receive the following ciphertext:
\[c=2400556132229818489305649515346654848298483477334619666591280284126769\]
Implement the quadratic sieve to factor $N$ and decrypt the message.

\solution
Who knows bro.
\end{document}
