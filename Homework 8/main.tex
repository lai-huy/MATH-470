\documentclass[12pt]{article}

\usepackage[affil-it]{authblk}
\usepackage[shortlabels]{enumitem}
\usepackage[utf8]{inputenc}
\usepackage{algorithm, algorithmicx, algpseudocode}
\usepackage{amsfonts, amsthm, amsmath, amssymb}
\usepackage{color}
\usepackage{cancel, textcomp}
\usepackage{enumerate}
\usepackage[mathscr]{euscript}
\usepackage{fancyhdr, fancyvrb}
\usepackage{fullpage}
\usepackage[left=0.5in,right=0.5in,headsep=0.5in,headheight=0.5in]{geometry}
\usepackage{graphicx}
\usepackage{hyperref}
\usepackage{latexsym}
\usepackage{mathtools}
\usepackage{minted}
\usepackage{times}
\usepackage{xcolor}
\usepackage{physics}
\usepackage{tikz-cd}
\usepackage[warnunknown, fasterrors, mathletters]{ucs}
\usepackage[nointegrals]{wasysym}

\newcommand{\hw}[2]{
    \noindent
    \begin{center}
        \framebox{
            \vbox{
                \hbox to 7in { {\bf MATH 470: Communications and Cryptography } \hfill  }
                \vspace{2mm}
                \hbox to 7in { {\Large \hfill Homework #1\hfill} }
                \vspace{2mm}
                \hbox to 7in { {\it Due date: #2 \hfill Name: Huy Lai } }
            }
        }
    \end{center}
    \vspace*{4mm}
}

\newcounter{prob}
\setcounter{prob}{0}
\newcounter{subprob}
\setcounter{subprob}{0}

\newcommand{\problem}{\setcounter{subprob}{0}\stepcounter{prob}{\noindent\textbf{Problem \theprob.}}\ }
\newcommand{\subproblem}{\stepcounter{subprob}{\noindent\textbf{Subproblem \thesubprob.}}\ }
\newcommand{\solution}{\noindent\textbf{Solution:}\newline}

\newcommand{\babc}{\begin{enumerate}[a)]}
\newcommand{\eabc}{\end{enumerate}}

\everymath{\displaystyle}

\setlength{\parskip}{.1in}
\setlength{\headheight}{15pt}
\setlength{\topmargin}{0pt}

\fancyhf{}
\pagestyle{fancy}
\lhead{MATH 470: Communications and Cryptography}
\rhead{Texas A\&M University}
\cfoot{\thepage}


\begin{document}
\thispagestyle{empty}
\hw{8}{25 October 2023}

\problem This exercise asks you to use the index calculus to solve a discrete logarithm problem. Let $p=19079$ and $g=17$\\
\subproblem Verify that $g^i\mod{p}$ is $5$-smooth for each of the values $i=3030,i=6892$ and $i=18312$.

\solution Calculate $g^i \mod p$
\begin{enumerate}
    \item For $i=3030$
          \begin{flalign*}
              g^{3030} \mod p & \equiv 17^{3030}\mod{p} & \\
                              & \equiv 14580\mod{p}     &
          \end{flalign*}
          $14580=2^2\cdot3^6\cdot5^1$

    \item For $i=6892$
          \begin{flalign*}
              g^{6892} \mod p & \equiv 17^{6892}\mod{p} & \\
                              & \equiv 18432\mod{p}     &
          \end{flalign*}
          $18432=2^{11}\cdot3^2$

    \item For $i=18312$
          \begin{flalign*}
              g^{18312}\mod{p} & \equiv 17^{18312}\mod{p} & \\
                               & \equiv 6000\mod{p}       &
          \end{flalign*}
          $6000=2^4\cdot3^1\cdot5^3$
\end{enumerate}

\newpage
\subproblem Use your computations in (a) and linear algebra to compute the discrete logarithms $\log_g(2)$, $\log_g(3)$ and $\log_g(5)$.

\solution
The discrete logs are as follows
\begin{flalign*}
    \log_g\left(g^{3030}\right)  & \equiv\log_p\left(2^2\cdot3^6\cdot5^1\right)\mod{p} & \\
    3030\log_g(g)                & \equiv2\log_g(2)+6\log_g(3)+\log_g(5)\mod{p}        & \\
    3030                         & \equiv2\log_g(2)+6\log_g(3)+\log_g(5)\mod{p}        & \\
    \log_g\left(g^{6892}\right)  & \equiv\log_p\left(2^{11}\cdot3^2\right)\mod{p}      & \\
    6892\log_g(g)                & \equiv11\log_g(2)+2\log_g(3)\mod{p}                 & \\
    6892                         & \equiv11\log_g(2)+2\log_g(3)\mod{p}                 & \\
    \log_g\left(g^{18312}\right) & \equiv\log_p\left(2^4\cdot3^1\cdot5^3\right)\mod{p} & \\
    18312\log_g(g)               & \equiv4\log_g(2)+1\log_g(3)+3\log_g(5)\mod{p}       & \\
    18312                        & \equiv4\log_g(2)+1\log_g(3)+3\log_g(5)\mod{p}
\end{flalign*}
Converting these congruences to matrixes is as follows:
\[
    \begin{bmatrix}
        2  & 6 & 1 \\
        11 & 2 & 0 \\
        4  & 1 & 3 \\
    \end{bmatrix}
    \begin{bmatrix}
        \log_g(2) \\
        \log_g(3) \\
        \log_g(5) \\
    \end{bmatrix}
    \equiv
    \begin{bmatrix}
        3030  \\
        6892  \\
        18312 \\
    \end{bmatrix}
    \mod{p}
\]
We use the note that $p-1=2\cdot9539$ and solve this linear system by splitting it $\mod{2}$ and $\mod{9539}$

\noindent
Calculating $\mod{2}$\\
$\left[
        \begin{array}{ccc|c}
            2  & 6 & 1 & 3030  \\
            11 & 2 & 0 & 6892  \\
            4  & 1 & 3 & 18312
        \end{array}
        \right]\equiv\left[
        \begin{array}{ccc|c}
            0 & 0 & 1 & 0 \\
            1 & 0 & 0 & 0 \\
            0 & 1 & 1 & 0
        \end{array}
        \right]\mod{2}$

\noindent
From this we get that
\[(x_2,x_3,x_5)\equiv(0,0,0)\mod{2}\]

\newpage
Calculating $\mod{9539}$\\
$\left[
        \begin{array}{ccc|c}
            2  & 6 & 1 & 3030  \\
            11 & 2 & 0 & 6892  \\
            4  & 1 & 3 & 18312
        \end{array}
        \right]\equiv\left[
        \begin{array}{ccc|c}
            2  & 6 & 1 & 3030 \\
            11 & 2 & 0 & 6892 \\
            4  & 1 & 3 & 8773
        \end{array}
        \right]\equiv\left[
        \begin{array}{ccc|c}
            1 & 3    & 4770 & 1515 \\
            1 & 8672 & 0    & 7564 \\
            1 & 2385 & 7155 & 4578
        \end{array}
        \right]\equiv\left[
        \begin{array}{ccc|c}
            1 & 3    & 4770 & 1515 \\
            0 & 6287 & 2384 & 2986 \\
            1 & 2385 & 7155 & 4578
        \end{array}
        \right]\equiv$\\
$\left[
        \begin{array}{ccc|c}
            1 & 3    & 4770 & 1515 \\
            0 & 6287 & 2384 & 2986 \\
            0 & 2382 & 2385 & 3063
        \end{array}
        \right]\equiv\left[
        \begin{array}{ccc|c}
            1 & 3    & 4770 & 1515 \\
            0 & 1523 & 7153 & 6399 \\
            0 & 2382 & 2385 & 3063
        \end{array}
        \right]\equiv\left[
        \begin{array}{ccc|c}
            1 & 3 & 4770 & 1515 \\
            0 & 1 & 8980 & 7564 \\
            0 & 1 & 5203 & 7558
        \end{array}
        \right]\equiv\left[
        \begin{array}{ccc|c}
            1 & 3 & 4770 & 1515 \\
            0 & 1 & 8980 & 7564 \\
            0 & 0 & 5762 & 9533
        \end{array}
        \right]\equiv$\\
$\left[
        \begin{array}{ccc|c}
            1 & 3 & 4770 & 1515 \\
            0 & 1 & 8980 & 7564 \\
            0 & 0 & 1    & 7463
        \end{array}
        \right]\equiv\left[
        \begin{array}{ccc|c}
            1 & 3 & 4770 & 1515 \\
            0 & 1 & 0    & 1299 \\
            0 & 0 & 1    & 7463
        \end{array}
        \right]\equiv\left[
        \begin{array}{ccc|c}
            1 & 3 & 0 & 1515 \\
            0 & 1 & 0 & 1299 \\
            0 & 0 & 1 & 7463
        \end{array}
        \right]\equiv\left[
        \begin{array}{ccc|c}
            1 & 0 & 0 & 8195 \\
            0 & 1 & 0 & 1299 \\
            0 & 0 & 1 & 7463
        \end{array}
        \right]$

\noindent
From this we get
\[(x_2,x_3,x_5)\equiv(8195,1299,7463)\mod{9539}\]

\noindent
Using the Chinese Remainder Theorem to combine these results:
\[(x_2,x_3,x_5)\equiv(17734,10838,17002)\mod{p}\]

\subproblem Verify that $19\cdot17^{-12400}\mod{p}$ is $5$-smooth.

\solution
We compute
\begin{flalign*}
    19\cdot17^{-12400} & \equiv 19\cdot\left(17^{-1}\right)^{12400}\mod{p} & \\
                       & \equiv 19\cdot(11223)^{12400}\mod{p}              & \\
                       & \equiv 19\cdot 5041\mod{p}                        & \\
                       & \equiv 384\mod{p}                                 & \\
                       & \equiv 2^7\cdot3\mod{p}
\end{flalign*}
This number is $5$-smooth

\newpage
\subproblem
Use the values from (b) and the computation in (c) to solve the discrete logarithm problem
\[17^x\equiv19\mod{p}\]

\solution
From part (c) we know that $19\equiv 17^{12400}\cdot(2^7\cdot3)\mod{p}$\\
Using the discrete logs we calculated in part (b) we can substitute $2$ and $3$ as follows
\begin{flalign*}
    19 & \equiv 17^{12400}\cdot(g^{17734})^7\cdot(g^{10838})^1\mod{p} & \\
    19 & \equiv 17^{147376}\mod{p}
\end{flalign*}
Note that $147376\equiv13830\mod{p-1}$\\
$x=13800$

\newpage
\problem Use the Tonelli-Shanks algorithm to compute a square root of $6$ modulo $97$ (note $97$ is prime).

\solution
$6^{\frac{p-1}{2}}\equiv6^{48}\equiv1\mod {97}$, therefore $6$ is a quadratic residue $\mod{97}$\\
$p-1 = 96 = 2^5\cdot3\rightarrow k=5,Q=3$\\
Let $z=5$ which is a non-quadratic residue $\mod {97}$

\noindent
Square root 6:\\
$6^3 \not\equiv 1 \mod {97}$\\
$i = 1$: $\;6^{2^i\cdot3} \equiv -1 \mod {97}$\\
$a' \equiv 6\cdot5^{2^{5-1-1}} \equiv 6\cdot5^{2^3} \equiv 36 \mod {97}$\\
$R = 91$\\
Return: $91 \cdot 5^{-2^{5-1-2}} \equiv 91 \cdot 5^{-2^2} \equiv 54 \mod {97}$

\noindent
Square root $36$ (Recursion 1):\\
$36^3 \not\equiv 1 \mod {97}$\\
$i = 0$: $\;36^{2^i\cdot3} \equiv -1 \mod {97}$\\
$a' \equiv 36\cdot5^{2^{5-0-1}} \equiv 36\cdot5^{2^4} \equiv 36 \mod {97}$\\
$R = 61$\\
Return: $61 \cdot 5^{-2^{5-0-2}} \equiv 61 \cdot 5^{-2^3} \equiv 91 \mod {97}$

\noindent
Square root 35 (Recursion 2):\\
$35^3 \equiv 1 \mod {97}$\\
Return: $35^2 \equiv 61 \mod {97}$

\noindent
$54$ is a square root of $6$ modulo $97$.

\newpage
\problem Using the fact that $2021=43\cdot47$ and that $43$ and $47$ are both primes, use the Tonelli-Shanks algorithm and the Chinese remainder theorem to compute a square root of $6\mod{2021}$.

\solution
Using the Tonelli-Shanks algorithm, we get that
\begin{flalign*}
    36^2 & \equiv 6\mod{43} & \\
    37^2 & \equiv 6\mod{47} &
\end{flalign*}
Calculate the modular inverses of 43 and 47 modulo each other
\begin{flalign*}
    43^{-1} & \equiv 35\mod{47} & \\
    47^{-1} & \equiv 11\mod{43}
\end{flalign*}
Next we use the Chinese Remainder Theorem to calculate
\begin{flalign*}
    x & \equiv 36\mod{43} & \\
    x & \equiv 37\mod{47} &
\end{flalign*}
From the first equivalence: $x=43y+36,y\in\mathbb{Z}$
\begin{flalign*}
    43y+36 & \equiv 37\mod{47} & \\
    43y    & \equiv 1\mod{47}  & \\
    y      & \equiv 35\mod{47} &
\end{flalign*}
$x=43(35)+36=1541$

\noindent
Therefore, $1541$ is a square root of $6\mod{2021}$\\
This can be verified by calculating that $1541^2\equiv6\mod{2021}$

\newpage
\problem Prove or disprove the following statement:\\
Let $N=pq$ where $p,q$ are distinct odd primes. If $a,b$ are integers such that $a^2\equiv b^2\mod{N}$ and $a\not\equiv\pm b\mod{N}$, then $\gcd(a-b,N)$ or $\gcd(a+b,N)$ gives a nontrivial factor $N$.

\solution This statement is true and the proof is as follows
\begin{proof}
    First we rewrite the given congruence $a^2\equiv b^2\mod{N}$ as
    \[(a-b)(a+b)\equiv0\mod{N}\]
    By proposition, this implies that $N=pq\mid(a-b)(a+b)$\\
    Since $p\mid N$, by proposition 1.4(a) and proposition 1.19, $p\mid(a-b)\lor p\mid(a+b)$.\\
    However, since $a\not\equiv\pm b\mod{N}$, $a+b\not\equiv0\mod{N}$ and $a-b\equiv0\mod{N}$.\\
    This means that neither $a-b$ nor $a+b$ is divisible by both $p$ and $q$.\\
    So one of $a-b$ or $a+b$ is divisible by $p$ but not by $q$, which means that $\gcd(a+b,N)=p$ or $\gcd(a-b,N)=p$
\end{proof}

\problem Read Example 3.69 in the textbook. Explain why $(-1)^{\text{dlog}_g(h)}=\left(\frac{h}{p}\right)$

\solution
If $\log_g(h)\equiv 0\mod{2}$, then $(-1)^{\log_g(h)}=1$\\
If $\log_g(h)\equiv 1\mod{2}$, then $(-1)^{\log_g(h)}=1$

\noindent
By proposition 3.61 we know that $h=g^{\log_g(h)}$ is a quadratic residue if and only if $\log_g(h)\equiv 0\mod{2}$ and that $h=g^{\log_g(h)}$ is a non-quadratic residue if $\log_g(h)\equiv1\mod{2}$

\noindent
By definition of the Legendre Symbol, $\left(\frac{h}{p}\right)=-1$ if $h$ is a non-quadratic residue modulo $p$ and $\left(\frac{h}{p}\right)=1$ if $h$ is.\\
From this the Legendre Symbol determines if $\log_g(h)$ is odd or even and we get the relationship described.

\noindent
The example 3.69 goes on to elaborate how because of this result, the $0$-th bit of the discrete logarithm in insecure.\\
What this means that $\left(\frac{h}{g}\right)$ can predict if the $0$-th bit is $0$ or $1$.

\newpage
\problem Let $p$ be an odd prime, let $g\in\mathbb{F}_p^*$ be a primitive root, and let $h\in\mathbb{F}_p^*$. Write $p-1=2^sm$ with $m$ odd and $s\geq 1$, and write the binary expansion of $\log_g(h)$ as
\[\log_g(h)=\epsilon_0+2\epsilon_1+4\epsilon_2+8\epsilon_3+\cdots\quad\text{with}\quad\epsilon_0,\epsilon_1,\cdots\in\{0,1\}\]
Give an algorithm that generalizes Example 3.69 and allows you to rapidly compute $\epsilon_0,\epsilon_1,\cdots\epsilon_{s-1}$, thereby proving that the first $s$ bits of the discrete logarithm are insecure. You may assume that you have a fast algorithm to compute square roots in $\mathbb{F}_p^*$, as provided for example by Exercise 3.39(a) if $p\equiv3\mod{4}$. (Hint. Use Example 3.69 to compute the 0th bit, take the square root of either $h$ or $g^{-1}h$, and repeat.)
\end{document}